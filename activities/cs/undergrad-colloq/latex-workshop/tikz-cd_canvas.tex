%---PREMABLE BEGINS HERE------------------------

\documentclass[11pt]{article}
%------------------------
%------------------------
%Packages
\usepackage[top=0.75in, bottom=1.25in, left=1in, right=1in]{geometry} 
\usepackage{amsmath,amsthm,amssymb} %this is THE math package
\usepackage{mathtools} %an extension to the amsmath package that helps you write more beautiful math
\usepackage{enumitem} %for better control of your lists
\usepackage{hyperref} %for hyperlinking stuff
%------------------------
%------------------------
\usepackage{tikz-cd}
%------------------------
\newcommand\tinydashv{\vcenter{\hbox{\scalebox{0.8}{$\dashv$}}}}
\tikzset{%
    symbol/.style={%
        draw=none,
        every to/.append style={%
            edge node={node [sloped, allow upside down, auto=false]{$#1$}}}
    }
}
%------------------------
%Useful packages for more specialised use
%\usepackage{tikz}
%\usepackage{graphicx}
%\usepackage{fancybox}
%\usepackage{varwidth}
%\usepackage{mdframed}
%\usepackage{mathrsfs}
%------------------------
\usepackage{xcolor} %the ultimate package for defining various colours
\definecolor{firebrick}{RGB}{178,34,34}
\definecolor{teal}{RGB}{0,128,128}
\definecolor{indigo}{RGB}{75,0,130}
\definecolor{lightgrey}{RGB}{212, 212, 212}
\definecolor{darkblue}{rgb}{0.0,0.0,.7}
\definecolor{darkred}{rgb}{0.7,0.0,0.0}
\definecolor{darkgreen}{rgb}{0.0,0.3,0.0}
\definecolor{transparent}{HTML}{EDEFF0}
\definecolor{white}{HTML}{FAFAFA}
\definecolor{bluegrey}{HTML}{34495E}
%------------------------
%------------------------
%Fonts I use, uncomment if you like to use them.
%The first is the general font, and the second a math font
%the third one is more specialised
\usepackage{mathpazo}
%\usepackage{eulervm}
\usepackage[scr=boondox]{mathalfa}
%------------------------
%------------------------
%This is so that we have standard fonts for the doublestroked symbols
%for reals, naturals etc. regardless of what font you use.
%Don't comment
\AtBeginDocument{
  \DeclareSymbolFont{AMSb}{U}{msb}{m}{n}
  \DeclareSymbolFontAlphabet{\mathbb}{AMSb}}
%------------------------
%------------------------
%Environments
\newtheorem{theorem}{Theorem}[section]
\newtheorem{lemma}[theorem]{Lemma}

\theoremstyle{definition}
\newtheorem{definition}[theorem]{Definition}

\newtheorem{problem}{Problem}[section]
%------------------------
%------------------------
%User-defined notations
\newcommand{\zz}{\mathbf Z}   %bold Z
\newcommand{\qq}{\mathbf Q}   %bold Q
\newcommand{\ff}{\mathbf F}   %bold F
\newcommand{\RR}{\mathbb R}   %blackboard bold R
\newcommand{\rr}{\mathbf R}   %bold R
\newcommand{\nn}{\mathbf N}   %bold N
\newcommand{\cc}{\mathbf C}   %bold C
\newcommand{\af}{\mathbf A}   %bold A
\newcommand{\pp}{\mathbf P}   %bold P
\newcommand{\id}{\operatorname{id}} %for identity map
\newcommand{\im}{\operatorname{im}} %for image of a function
\newcommand{\dom}{\operatorname{dom}} %for domain of a function
\newcommand{\cok}{\operatorname{coker}} %cokernel
\newcommand{\cat}[1]{\mathsf{#1}}
\newcommand{\abs}[1]{\left\lvert#1\right\rvert} %for absolute value
\newcommand{\norm}[1]{\left\lVert#1\right\rVert} %for norm
\newcommand{\modar}[1]{\operatorname{mod}{#1}} %for modular arithmetic
\newcommand{\set}[1]{\left\{#1\right\}} %for set
\newcommand{\setp}[2]{\left\{#1\ \middle|\ #2\right\}} %for set with a property
%------------------------
%------------------------
%Re-defined notations
\renewcommand{\epsilon}{\varepsilon}
\renewcommand{\displaystyle}{\disp}
%\renewcommand{\phi}{\varphi}
\renewcommand{\emptyset}{\varnothing}
\renewcommand{\geq}{\geqslant}
\renewcommand{\leq}{\leqslant}
\renewcommand{\Re}{\operatorname{Re}}
\renewcommand{\gcd}{\operatorname{GCD}}
\renewcommand{\Im}{\operatorname{Im}}
%\renewcommand{\qedsymbol}{$\blacksquare$}
%------------------------
%------------------------
\allowdisplaybreaks
\setlength\parindent{0pt} %controls indentation after paragraph break
%------------------------
%------------------------

%---PREAMBLE ENDS HERE------------------------


%---DOCUMENT BEGINS HERE------------------------
\begin{document}
 
\title{\texttt{tikz-cd} Canvas}
\author{You}
\date{\today} 
\maketitle

%Use \[...\] instead of $$...$$

We add \verb!\usepackage{tikz-cd}! to the Preamble. Consider the following figures. Investigate the code, tinker with it and see what pieces of the code does what.
\[\begin{tikzcd}[column sep=scriptsize]
\cat{C} \arrow[rr, "F"', shift right=2,""{name=A1,above}] \arrow[dd, "\Delta", shift left=2,""{name=C1,below}] &  & \cat{D} \arrow[dd, "\Delta"', shift right=2,""{name=D1,above}] \arrow[ll, "G"', shift right=2,""{name=A2,below}]\ar[from=A1, to=A2, symbol=\tinydashv] \\
                                                                           &  &                                                                              \\
\cat{C^J} \arrow[rr, "F_*", shift left=2,""{name=B2,below}] \arrow[uu, "\lim", shift left=2,""{name=C2,below}]\ar[from=C1, to=C2, symbol=\tinydashv, shift right=1] &  & \cat{D^J} \arrow[uu, "\lim"', shift right=2,""{name=D2,above}] \arrow[ll, "G_*", shift left=2,""{name=B1,above}]\ar[from=B2, to=B1, symbol=\tinydashv]\ar[from=D1, to=D2, symbol=\tinydashv, shift right=1]
\end{tikzcd}\]
\\
\[\begin{tikzcd}[row sep=huge]
  & {\color{darkgreen}\ker\alpha} \rar[darkgreen] \arrow[d,hook] & {\color{darkgreen}\ker\beta} \rar[darkgreen] \arrow[d,hook] & {\color{darkgreen}\ker\gamma} \arrow[d,hook] \ar[out=0, in=180, darkgreen]{dddll} & \\
  & M'\arrow[d,"\alpha" description] \rar & M \rar\arrow[d,"\beta" description] & M'' \rar\arrow[d,"\gamma" description] & 0\\
  0 \rar & N' \rar \arrow[d,two heads] & N \rar \arrow[d,two heads] & N'' \arrow[d,two heads] & \\
   & {\color{darkgreen}\cok \alpha} \rar[darkgreen] & {\color{darkgreen}\cok \beta} \rar[darkgreen] & {\color{darkgreen}\cok \gamma} &
\end{tikzcd}\]
\\
The way to generate the less commutative diagrams is to head to \url{https://q.uiver.app/} and \url{https://tikzcd.yichuanshen.de/}. Go and experiment!\\
\\
For this session, to begin with, try drawing the figure on the blackboard.

%---DOCUMENT ENDS HERE------------------------
\end{document}
